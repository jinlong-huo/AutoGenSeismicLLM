\documentclass[10pt]{article}

\usepackage{fullpage}
\usepackage{CJKutf8} 
\usepackage[utf8]{inputenc}
\usepackage{setspace}
\usepackage{parskip}
\usepackage{titlesec}
\usepackage[section]{placeins}
\usepackage{xcolor}
\usepackage{breakcites}
\usepackage{lineno}
\usepackage{hyphenat}
% \usepackage{xcolor}




\PassOptionsToPackage{hyphens}{url}
\usepackage[colorlinks = true,
            linkcolor = blue,
            urlcolor  = blue,
            citecolor = blue,
            anchorcolor = blue]{hyperref}
\usepackage{etoolbox}
\makeatletter
\patchcmd\@combinedblfloats{\box\@outputbox}{\unvbox\@outputbox}{}{%
  \errmessage{\noexpand\@combinedblfloats could not be patched}%
}%
\makeatother


\usepackage{natbib}




\renewenvironment{abstract}
  {{\bfseries\noindent{\abstractname}\par\nobreak}\footnotesize}
  {\bigskip}

\titlespacing{\section}{0pt}{*3}{*1}
\titlespacing{\subsection}{0pt}{*2}{*0.5}
\titlespacing{\subsubsection}{0pt}{*1.5}{0pt}


\usepackage{authblk}


\usepackage{graphicx}
\usepackage[space]{grffile}
\usepackage{latexsym}
\usepackage{textcomp}
\usepackage{longtable}
\usepackage{tabulary}
\usepackage{booktabs,array,multirow}
\usepackage{amsfonts,amsmath,amssymb}
\providecommand\citet{\cite}
\providecommand\citep{\cite}
\providecommand\citealt{\cite}
% You can conditionalize code for latexml or normal latex using this.
\newif\iflatexml\latexmlfalse
\providecommand{\tightlist}{\setlength{\itemsep}{0pt}\setlength{\parskip}{0pt}}%

\AtBeginDocument{\DeclareGraphicsExtensions{.pdf,.PDF,.eps,.EPS,.png,.PNG,.tif,.TIF,.jpg,.JPG,.jpeg,.JPEG}}

\usepackage[utf8]{inputenc}
\usepackage[english]{babel}










\begin{document}
\begin{CJK}{UTF8}{gbsn}

\title{Nature Communications Template}



\author[1]{Jyoti Chandra}%
\affil[1]{Affiliation not available}%


\vspace{-1em}



  
  \date{May 27, 2025}


\begingroup
\let\center\flushleft
\let\endcenter\endflushleft
\maketitle
\endgroup





\selectlanguage{english}
\begin{abstract}
{This Authorea document template can be used to prepare documents
according to a desired citation style and authoring guidelines.
Abstracts are not always required, but most academic papers have one and
writers should know how to produce a useful abstract. An abstract should
be a very short, clear and concise summation of the entire paper. An
abstract should provide enough of a preview that a typical reader will
know whether or not they wish to read the paper. It should reveal both
the purpose and conclusions of the paper.}\\%
\end{abstract}%



\sloppy


This Authorea document template can be used to prepare documents
according to a desired citation style and authoring guidelines.
Abstracts are not always required, but most academic papers have one and
writers should know how to produce a useful abstract. An abstract should
be a very short, clear and concise summation of the entire paper.

An abstract should provide enough of a preview that a typical reader
will know whether or not they wish to read the paper. It should reveal
both the purpose and conclusions of the paper.

\par\null

\section*{Introduction}\label{auto-label-section-226136}

The format of this template follows a typical journal publication with
an \textbf{introduction}, \textbf{results} and \textbf{conclusion}.
Examples of an equation, list and citation are also included.~

\par\null

\subsection*{The purpose of the
introduction}\label{auto-label-subsection-616396}

Most academic introductions follow an `inverted pyramid' structure: they
start broad and narrow down to a specific thesis or research question.
The introduction should reveal:

\begin{enumerate}
\tightlist
\item
  some broad knowledge of the overall topic
\item
  references to related and prior work in the field of investigation
\item
  succinct overview of the major point of the paper
\end{enumerate}

\par\null

\section*{Results}\label{auto-label-section-330806}

This section is only included in papers that rely on primary research.
This section catalogues the results of the experiment. The results
should be presented in a clear and unbiased way. Most results sections
will contain \href{https://authorea.com}{links}~and citations,
e.g.,~\textsuperscript{\hyperref[csl:1]{1}}, and equations, e.g.~\(e^{i\pi}+1=0\).

\par\null

\section*{Conclusion}\label{auto-label-section-226071}

The conclusion should reinforce the major claims or interpretation in a
way that is not mere summary. The writer should try to indicate the
significance of the major claim/interpretation beyond the scope of the
paper but within the parameters of the field. The writer might also
present complications the study illustrates or suggest further research
the study indicates is necessary.

\selectlanguage{english}
\FloatBarrier
\section*{References}\sloppy
\phantomsection
\label{csl:1}1.Feynman, R. P. \& Teich, M. C. {Surely You{\textquotesingle}re Joking Mr. Feynman! Adventures of a Curious Character}. \textit{Physics Today} \textbf{39}, 61–61 (1986).

\end{CJK}\end{document}

